%\documentclass[12pt,conference,a4paper,onecolumn]{book}%{IEEEtran} % altura de letra=10, conferencia, formato a4. "`argencon"' o "`IEEEtran"' es la clase del monumento, que si no es el estandar, tiene sus propias opciones, como "`conference"'.journal
%\usepackage[lmargin=2.5cm,rmargin=1.5cm,top=1.5cm,bottom=1.5cm]{geometry} % es para acomodar los margenes
%-------------Configuracion de la UBA
%\documentclass[12pt,a4paper,twoside]{tesis}
% SI NO PENSAS IMPRIMIRLO EN FORMATO LIBRO PODES USAR
%\documentclass[11pt,a4paper]{tesis}
\usepackage[left=2.5cm,right=2cm,bottom=2cm,top=2cm]{geometry}
%-------------Configuracion de la UBA
%\usepackage{mathptmx} % Timer New Roman font
%\renewcommand{\baselinestretch}{1.5} % Interlineado


% Para comentar un codigo rapido es con ctrl+q y para descomendar es con ctrl+w
%\newcommand{\CLASSINPUTbottomtextmargin}{10mm} % redefino el margen inferior
%\usepackage[latin1]{inputenc} %acentos sin codigo extra
%\usepackage[cp1252]{inputenc}% utf8

\usepackage[compress]{cite} %Es para agregar la bibliografia al final
%\usepackage[compress]{natbib} % esta se usa en todos lados. Pero no vale par IEEE
%\usepackage[pdftex]{graphicx} % PDFLaTeX
%\usepackage[dvipdfmx]{graphicx} % PDFLaTeX

%\usepackage[dvips]{graphicx} % LaTeX
%\DeclareGraphicsExtensions{.eps}

\usepackage{graphicx}%es para agregar imagenes 
\usepackage{float} % para que las imagenes entren como flotantes con [H]
%\usepackage[pdftex]{graphicx}%es para agregar imagenes % Tira un error
%\usepackage{subfigure}  %es para las subfiguras
\usepackage{subcaption} % poner mejores subfiguras?
\usepackage{booktabs}
\usepackage[table,xcdraw]{xcolor}
\usepackage{placeins}

%\DeclareGraphicsExtensions{.eps,.png}
%\DeclareGraphicsExtensions{.png,.eps}
%\usepackage[spanish]{babel} % es para castellanizar algunos comandos
%\usepackage[spanish, es-tabla]{babel} % para que en vez de cuadro diga tabla
\usepackage[spanish,es-noshorthands,es-tabla]{babel} %agrega unos caracteres extra al spanish de babel
\usepackage{alltt} %Ver para que es esto, me parece que simplifica el "` vervatim"'
\usepackage{listings} % es para poder cargar los codigos y que se vean bonitos
\renewcommand\lstlistingname{Código} % Sirve para cambiar el titulo de los codigos. En vez de decir "listing"
%\usepackage{eps}  
\usepackage{chngcntr} % para los pies de pagina
\usepackage[hidelinks]{hyperref} % para agregar links "`[hidelinks]"' sirve para sacar el recuadro alrededor del link
\usepackage{amsthm} % ambiente matematico
\usepackage{amsmath} %otro ambiente matematico
\usepackage{amsfonts} % Para agregar las letras de los conjuntos
\usepackage{mathrsfs} % Letras de transformadas
\usepackage{mathtools}%otro ambiente matematico
\usepackage{amssymb} % mas simbolos
%\usepackage{slashbox} % Es para poner la linea en la tabla
\usepackage[compress]{cite} %Es para agregar la bibliografia al final
%\usepackage{hyperref} % para poner un indice con que linkee a todo el texto en el pdf
\usepackage{color} % para poner textos con color 
\usepackage{hyperref} % para poner links
%\usepackage{epsfig} % para poner color en las tablas
\usepackage{multirow}% para poner color en las tablas
\usepackage{colortbl}% para poner color en las tablas. Esta es la mas importante 
%\columncolor[Modelo]{Color}[SepIzq][SepDer] (columnas) \rowcolor[Modelo]{Color}[SepIzq][SepDer] (filas) 
% http://metodos.fam.cie.uva.es/~latex/apuntes/apuntes10.pdf
\usepackage[table]{xcolor}% para poner color en las tablas % http://ctan.org/pkg/xcolor
\usepackage{array} % Para corregir el alineamiento en las tablas
\usepackage{adjustbox} % Para que las tablas no se salgan de los margenes
\usepackage{fancyhdr} % Es para el encabezado y el pie de pagina
\usepackage{stackengine} % darle un espacio a las filas en ambiente tabular
%\hline\xrowht[()]{10pt}
%\usepackage{TeXiS/TeXiS}% Paquete de la plantilla
%\usepackage{blindtext}
\usepackage{enumitem} % Es para cambiar la forma de enumerar:  
% Letras minusculas
%\begin{enumerate}[label=(\alph*)]
% Letras mayusculas
%\begin{enumerate}[label=(\Alph*)]
% Letras romanas
%\begin{enumerate}[label=(\roman*)]
\usepackage{textcomp} % Es para agregar las marcas registradas
% \textregistered\textcopyright
%\sffamily\textregistered\textcopyright
\usepackage{pgfgantt} % Para poner diagramas de gantt
% ------------------- \newcommand
\newcommand{\simulink}{Simulink\textsuperscript{\textregistered}\ }
\newcommand{\matlab}{MATLAB\textsuperscript{\textregistered}\ }
%\renewcommand{\tablename}{Tabla} % para que diga tabla en vez de ``cuadro''
%\newcommand{\be}{\begin{equation}}  %Simplifica la escritura de ecuaciones
%\newcommand{\ee}{\end{equation}}  %Simplifica la escritura de ecuaciones
%\newcommand{\nbe}{\begin{equation*}}  %Simplifica la escritura de ecuaciones
%\newcommand{\nee}{\end{equation*}}  %Simplifica la escritura de ecuaciones
%\newcommand{\m}{\begin{bmatrix}}
%\newcommand{\fm}{\end{bmatrix}}
%\newcommand{\fig}[4]{ % Figuras asi se usa: \fig{Nombrefigura}{epigrafe}{donde? h, t, b, htb}{ancho}
%\begin{figure}[#3]
%\centering
%\includegraphics[width=#4\columnwidth]{./Figuras/#1}
%\caption{#2}\label{fig:#1}
%\end{figure}}
%
%\newcommand{\arr}[2]{ % es para poder apilar ecuaciones NO numeradas
%\begin{equation*}
%\begin{array}{#1} #2 
%\end{array}
%\end{equation*}}
%
%\newcommand{\pap}{paso a paso\  }
%\newcommand{\Mic}{microstepping}
%\newcommand{\xx}{\cellcolor{blue!50}}


% --------------- FIN \newcommand ----------------------
% Para poner un lindo codigo de matlab:
% For faster processing, load Matlab syntax for listings
\definecolor{codegreen}{rgb}{0,0.6,0}
\definecolor{codegray}{rgb}{0.5,0.5,0.5}
\definecolor{codepurple}{rgb}{0.58,0,0.82}
\definecolor{backcolour}{rgb}{0.95,0.95,0.92}
 
\lstdefinestyle{mystyle}{
    backgroundcolor=\color{backcolour},   
    commentstyle=\color{codegreen},
    keywordstyle=\color{magenta},
    numberstyle=\tiny\color{codegray},
    stringstyle=\color{codepurple},
    basicstyle=\ttfamily\footnotesize,
    breakatwhitespace=true,         
    breaklines=true,                 
    captionpos=b,                    
    keepspaces=false,                 
    numbers=left,                    
    numbersep=5pt,                  
    showspaces=false,                
    showstringspaces=false,
    showtabs=false,                  
    tabsize=2
}
 
\lstset{style=mystyle}
  %---------------------------------------------------------
  
  
%Agrego dependencias para el book

